\section{Softwareimplementierung}
\subsection{Betriebssystem und Entwicklungsumgebung}
\paragraph{}
\subsection{Softwarearchitektur}
\subsubsection{Single Application vs. ROS}
\paragraph{}
Wie es häufig in der Entwicklung der Fall ist, stehen auch hier für die Implementierung
der Software auf dem RPi verschiedene Plattformen, Frameworks und
Programmiersprachen zu Verfügung. Bei der Programmiersprache viel die Wahl recht
schnell auf Python, da es einem im Vergleich zu C/C++, aber auch Java viel Overhead
abnimmt, in der allgemeinen Komplexität geringer ist und vielerorts (wie auch in ROS,
siehe unten) nativ unterstützt wird.
\paragraph{}
Bei der Softwarearchitektur wurde sich zunächst an einer normalen
objektorientierten Struktur bedient: Es gibt eine main-Datei, welche vergleichsweise
klein ist und nur die grobe Struktur bzw. den Gesamtablauf darstellt. Diese ruft dann
weitere Softwaremodule auf, die die Unteraufgaben, wie beispielsweise das
Empfangen und Verarbeiten von Bilddaten, implementiert. Ein Problem, das hier recht
schnell auftritt ist, dass theoretisch zwei Prozesse gleichzeitig ablaufen müssen. Wenn
ein Motormodul beispielsweise gerade darauf wartet, dass die Sollposition erreicht
wurde, müsste in einem Programmierstrang, in dem ein Befehl nach dem anderen
ausgeführt wird, das Kameramodul mit der Verarbeitung des Bildes auf das
Motormodul warten, obwohl das Motormodul gerade gar nichts tut, außer auf den
Motor in der echten Welt zu warten, bis er die richtige Position hat. Da das
Kameramodul aber an das Motormodul eventuell kommunizieren muss, dass das
Paket gar nicht mehr an der richtigen Stelle ist, weil der Quadrokopter aufgrund
äußerer Umstände nicht ganz stillsteht, müssen beide Module gleichzeitig arbeiten und
miteinander kommunizieren können. In der traditionellen Softwareentwicklung gibt es
dafür sogenannte Threads. Diese sind separate Ausführungsstränge, die gleichzeitig
ausgeführt werden und sich dann beispielsweise ein Datenobjekt im Speicher teilen
und über dieses miteinander kommunizieren können. Das Problem bei dieser Sache:
Multithreading (mehrere Threads) ist eine recht komplexe Angelegenheit, da sehr
vielseitige Probleme auftreten können. Will zum Beispiel Prozess A auf die Variable
Paketposition im Datenobjekt zugreifen und Prozess B will diese gleichzeitig
beschreiben, wird das Programm abstürzen.
\paragraph{}
Bis zu einem gewissen Grad war es möglich die Funktionalitäten mit Multithreading zu
implementieren. Allerdings wurde es im Laufe der Entwicklung immer komplizierter, da
Bilddaten beispielsweise von zwei Ressourcen abhängen, die nicht immer verfügbar
sind: zum einen der Speicherplatz im Datenobjekt, das andere Threads lesen und
damit belegen wollen und zum anderen von der Kamera selbst, die natürlich auch nicht
willkürlich Bilddaten liefert. Deshalb basiert die endgültige Software auf ROS, das
genau diese Probleme adressiert. ROS steht dabei für Robot Operating System und
ist ein Framework, das auf dem Publisher-Subscriber-Modell basiert. Die Idee ist
dabei, dass es verschiedene Themen (Topics) gibt, in die verschiedene dezentrale
Skripte, die gleichzeitig laufen, Informationen zu einzelnen Themen veröffentlichen
oder abonnieren können. So gibt es beispielsweise ein Kameraskript, das das Bild und
die erkannte Position des Pakets veröffentlicht und ein Motorskript abonniert die
Paketposition und kann daraus dann weitere Schlüsse ziehen. Ein anderes Skript
könnte dann auf einem anderen Rechner (z.B. Laptop) im gleichen (WLAN-) Netzwerk
das Kamerabild abbonnieren und mit sehr wenig Programmieraufwand anzeigen. Das
elegante dabei ist, dass sämtliche vorher beschriebene Multithreading-Probleme dabei
vom sogenannten ROS-Core, der zentrale Schaltstelle des ROS, übernommen
werden und damit die einzelnen (selbstgeschriebenen) Skripte sehr entschlackt
werden. Das macht es deutlich einfacher diese zu debuggen. ROS ist zudem sehr gut
dokumentiert und für viele Probleme gibt es bereits vorgefertigte Skripte, die Dank des
einfachen P/S-Modells auch einfach anzubinden sind.

\subsection{Robot Operating System (ROS)}
\subsubsection{Mavros und Simulation}
\subsection{Dronensteuerung}