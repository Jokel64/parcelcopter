%###########################################################################
%
%   Anhang
%
%###########################################################################
\begin{appendix}
\chapter*{Anhang}
\setcounter{chapter}{1}
\addcontentsline{toc}{chapter}{Anhang}

%###########################################################################
%   Anhang A
%###########################################################################
\section{Überschrift von Anhang A}
\markboth{Anhang}{Anhang}
Text

%###########################################################################
%    Anhang B
%###########################################################################
\section{Überschrift von Anhang B}
\markboth{Anhang}{Anhang}
Text

%###########################################################################
%    Anhang C (entfaellt, wenn keine für diese Arbeit relevanten Daten, 
%    Hilfsprogrammen, Skripts und Simulationsumgebungenabzulegen sind)
%###########################################################################
\pagebreak
\section{Inhalt der CD-ROM}
\markboth{Anhang}{Anhang}
%
% {XXX} durch Studienarbeitsnummer ersetzen!!!
%
Die beigelegte CD-ROM enthält in der obersten Dateistruktur die Einträge
\begin{itemize}
\item \textbf{stud$\_${XXX}.pdf}: das PDF-File zur Studienarbeit
STUD--{XXX}.
%
\item \textbf{STUD$\_${XXX}/}: ein Verzeichnis mit den TEX-Dateien des in
LaTeX verfassten Berichtes zur Studienarbeit STUD--{XXX} sowie alle
dazugehörigen Grafiken als *.eps und *.svg Dateien.
% 
\item \textbf{DATA/}: ein Verzeichnis mit den für diese Arbeit
relevanten Daten, Hilfsprogrammen, Skripts und Simulationsumgebungen.
%
\end{itemize} 
Zusätzliche Informationen stehen in den readme.txt-Dateien der
jeweiligen Verzeichnisse zur Verfügung.

%###########################################################################
\end{appendix}
