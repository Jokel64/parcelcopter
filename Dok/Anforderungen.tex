\section{Anforderungen}
\subsection{Softwareanforderung}
Ein weiterer Teil des Projektes ist die Software. Die Aufgabe dieser erstreckt sich von der Bildverarbeitung über den Objekterkennungsprozess bis hin zu der Regelung und Steuerung der Drohne. Dabei muss sie verschiedensten Kriterien genügen.\\
\\
Da sie Signale von einem Externen Bauteilen wie dem PX4 oder dem Abstandssensor über das Interface ROS(siehe Kapitel Interface) kann es zu Datenverlust kommen. Bedingt durch die unterschiedlichen Taktrate der Prozessoren. Deshalb muss auch bei gelegentlich fehlenden Datenpaketen der Rest verarbeitet und die Drohne auf Grundlage dieser geregelt werden.\\
\\
Der PX4-Controller schreibt eine Datenrate von 20 Hz vor. Also muss in dieser Datenrate die Sollposition der Drohne geschickt werden. Fallen diese aus, so geht die Drohne in den Offboard-Modus und steigt auf 10 m Höhe, um dann automatisch zur Startposition zurückzukehren. Dies wäre bei einer Raumhöhe von 2,5 m fatal. Leider kann man diese „Schutzfunktion“ nicht abschalten da sie fest im Betriebssystem des PX4-Controllers verankert ist. Demzufolge darf das Signal der Position nie ausbleiben.\\
\\
Um dies zu erreichen, darf sich die Regeleinheit der Drohne nie aufhängen. Alle Schleifen müssen deshalb ein zeitliches Abbruchsignal haben, um eine Auslastung des Arbeitsspeichers zu verhindern. Aufgrund des Gewichtes muss ein leichter Controller gewählt werden(siehe Hardware Komponenten). Da wir aufgrund der Software Inkompatibilität des Pi-4 mit dem PX4-Controller einen Pi-3 gewählt haben steht ein maximaler Arbeitsspeicher von 1 Gb zur Verfügung. Dabei hat er 2 Kerne und kann somit effektiv 2 Aufgaben gleichzeitig ausführen. Die Software muss also sehr ressourcensparend geschrieben sein.\\
\\
Aufgrund der Struktur von ROS empfiehlt es sich, einen Event basierten Programmablauf zu schreiben. Dabei löst ein Event, also eine Aktion die stattfindet, einen Programmablauf aus. In diesem Fall sind die Events eingehende Datenpakete für die entsprechende Node (Siehe Programmstruktur). Diese werden verarbeitet. Dadurch kann sichergestellt werden, dass jedes Datenpaket verarbeitet wird, unabhängig in welchem Abstand es ankommt. Durch diese Warte-Bedien-Struktur wird eine maximale Effizienz erreicht, da das Programm keine Schritte doppelt auf nicht aktualisierte Werte anwendet und so Prozessorzeit spart.