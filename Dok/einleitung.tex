\chapter{Einleitung}
Bei dieser Projektarbeit geht es um die Entwicklung einer Paketgreiffunktion für einen
bereits funktionierenden Quadrokopter. Die zu entwickelnde Hardware lässt sich dabei
in vier Kategorien einteilen: Ein Greifarm, der in der Lage ist, Pakete mit einer Größe
von ca. 10 cm zu greifen und zu halten; eine Kamera, die das zu greifende Paket
erkennt; diverse Sensorik und Aktorik für den Greifarm und anderweitige
Kontrollfunktionen; ein System-on-a-Chip (SoC) für die Bildanalyse, Regelung und
Steuerung des Greifarms und des Quadrokopters, wobei dieser selbst bereits mit
einem weiteren SoC ausgestattet ist, mit welchem für die Steuerung nur kommuniziert
werden muss. Es gibt eine Vielzahl verschiedener SoCs, wobei sich hier aufgrund
weiter Verbreitung für den Raspberry Pi (RPi) entschieden wurde, welcher, salopp
gesagt, als das Gehirn der neu entwickelten Funktionalität dient. Im Folgenden soll es
nun um die grundlegende Softwarearchitektur und -infrastruktur gehen, welche die
Aufgaben des Raspberry Pis auf diesem implementiert.