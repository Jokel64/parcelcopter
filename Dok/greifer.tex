\section{Greifermechanik}
\subsection{Einleitung}
In Zeiten wo manchmal jede Sekunde zählt und der Verkehrsfluss auf den Straßen
nicht immer garantiert ist, ist der schnelle Luftweg mit einer Drohne oftmals die bessere
Option. Ein Automatisierter Transport über den Luftverkehr könnte die Lebensqualität
in vielen Bereichen verbessern und dabei Kosten und Umweltschadstoffe, durch die
ersparte Fahrt, sparen. Dabei geht der konstruktive Aspekt, nicht nur aber vor allem,
im Luftverkehr mit Leichtbau und gleichzeitiger Stabilität einher.
Ziel der konstruktiven Arbeit am und mit dem Quadrokopter ist ein modular
montierbarer Greifarm, welcher mit Kameras und Sensoren sein Ziel selbst finden
kann. Dabei ist die Mechanik des Greifarms der letztendliche Kern der Arbeit und
beginnt an der Verbindungsstelle mit dem Motor und endet mit der letztendlichen
Kontaktstelle zur Last, welche transportiert werden soll.
Dabei sollte der Fokus der konstruktiven Ausarbeitung eine stabile, funktionssichere,
leichte und dauerfeste Konstruktion sein, welche selbst bei nicht idealen Bedingungen
ihre Aufgaben erfüllt und dabei insbesondere das Wohlergehen von Passanten nicht
gefährdet. Beim Aspekt der Stabilität und Sicherheit ist insbesondere die
Greifsicherung im Falle eines Defekts und die Lage des Schwerpunkts zu beachten.
Gleichzeitig darf ein maximales Gewicht von 2kg nicht überschritten werden, da sonst
ein spezieller Drohnenführerschein als Nachweis zum Bedienen des Quadrokopters
nötig ist.
Für die Konstruktion steht als Grundmaterial für den Rahmen Plexiglas zur Verfügung,
welches mit Laserschneidemaschinen in Formen geschnitten werden kann. Weitere
benötigte Bauteile wie zum Beispiel ein Motor oder Lagerungen gelten als Zukaufteile.
Die tiefere Forschungsfrage hinter dem Projekt ist die Visualisierung des Greifarms im
späteren Verlauf, welche Möglichkeiten einem bei der Konstruktion beim Leichtbau
und der Stabilität mit dem vorausgesetztem Material gegeben sind und wie gut das
Konzept umsetzbar ist.
Für die Auswahl eines Konzepts wird zwischen mehreren Prinzipskizzen eine
kategorisch ausgewählt und vollständig dimensioniert. Dann wird nochmals die
Realisierung geprüft mit einem CAD Modell bzw. Simulation. 